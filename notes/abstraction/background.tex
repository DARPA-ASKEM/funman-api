\begin{definition}
    A Petrinet $\Omega$ is a directed graph $(V, E)$ with vertices $V=(V_x,
    V_z)$ partitioned into sets $V_x$ of state vertices and $V_z$ of transition
    vertices, and edges $E=(E_{in}, E_{out})$ partitioned into sets $E_{out}$ of
    flow-out and $E_{in}$ flow-in edges. 
\end{definition}

\begin{definition}
A flow-out edge $e \in E_{out}$ comprises a pair of vertices $(v_x,v_z)$, where
$v_x \in V_x$ is a state vertex, $v_z \in V_z$ is a transition vertex, and the
flow is directed from $v_x$ to $v_z$.  
\end{definition}

\begin{definition}
    A flow-in edge $e \in E_{in}$ comprises a pair of vertices $(v_z,v_x)$,
    similar to a flow-out edge, except that the flow is directed from $v_z$ to
    $v_x$.  
\end{definition}



\begin{definition}
    The ODE semantics $\Theta$ of the Petrinet $\Omega$ defines a tuple $(P, X,
    Z, {\cal I}, {\cal P}, {\cal X}, {\cal Z}, {\cal R})$ where 
    \begin{itemize}
        \item $P$ is a set of parameters;
        \item $X$ is a set of state variables;
        \item $Z$ is a set of transitions;
        \item ${\cal I}: S \rightarrow \reals$ assigns the initial value of
        state variables to a real number;
        \item ${\cal P}: P \rightarrow \reals \cup \reals \times \reals$ assigns
        parameters to a real number, or a pair of real numbers defining an
        interval;
        \item ${\cal X}: X \rightarrow V_x$ assigns state variables to state
        vertices;
        \item ${\cal Z}: Z \rightarrow V_z$ assigns transtions to transition
        vertices; and
        \item ${\cal R}: {\bf P} \times {\bf X} \times Z \rightarrow \reals$
        defines the rate of each transition in $x \in X$ in terms of the set of
        parameter vectors ${\bf P}$ and state variable vectors ${\bf X}$.  
    \end{itemize}
    The elements of the Petrinet $\Omega$ and semantics $\Theta$ define the
    partial derivative $\frac{d {\bf x}}{dt}$, so that for each state variable
    $x \in X$:
    
    \begin{equation}\label{eqn:flow}
        \frac{dx}{dt} = \sum_{v_z \in V_z^{in(x)}} {\cal R}({\bf p}, {\bf x}, z) - \sum_{v_z \in V_z^{out(x)} } {\cal R}({\bf p}, {\bf x}, z)
    \end{equation}
\noindent where $V_z^{in(x)} = \{v_z \in V_z | (v_z, v_x) \in E_{in}\}$ and
    $V_z^{out(x)}=\{v_z \in V_z| (v_x, v_z) \in E_{out}\}$ are the transition
    vertices that flow in and out of the vertex $v_x$, respectively. We denote
    by $\nabla_{\Omega, \Theta}({\bf p}, {\bf x}, t) = (\frac{dx_1}{dt},
    \frac{dx_2}{dt}, \ldots)^T$, the gradient comprised of components in
    Equation \eqref{eqn:flow}.
\end{definition}

Using the partial derivatives defined by the Petrinet graph and semantics, we
can define the state vector at given time $t+dt$ with the forward Euler method
as:

\begin{eqnarray*}
    \frac{d {\bf x}}{dt} &=& \nabla_{\Omega, \Theta}({\bf p},{\bf x}, t)\\
    \frac{{\bf x}(t+dt)-{\bf x}(t)}{dt} &=& f_{\Omega, \Theta}({\bf p},{\bf x},
    t)\\
    {\bf x}(t+dt)&=& f_{\Omega, \Theta}({\bf p},{\bf x}, t)dt+ {\bf x}(t)
\end{eqnarray*}

\begin{definition}
    An abstraction $(\Theta', \Omega')$ of a Petrinet and the associated
    semantics $(\Theta, \Omega)$ that is produced by the abstraction operator
    $A$ has the following properties:
    \begin{itemize}
        \item State: For each $x \in X$,  $A(x) = x'$, where $x' \in
        X'$.  For each vertex $v_x \in V_x$,  $A(v_x) = v_x'$ where $v_x' \in
        V_x'$.   For each $x\in X$ where  ${\cal X}(x) =
        V_x$, $A(x) = x'$, and $A(v_x) = v_x'$, then ${\cal X}'(x')=
        v_{x'}'$.  For each $x' \in X'$, ${\cal X}'(x') = \sum\limits_{x \in X: A(x) = x'} {\cal X}(x)$.
        \item Parameters: For each $p \in P$, $A(p) = p'$, where $p'\in P'$.
        For each $p' \in P'$, ${\cal P}'(p') = \sum\limits_{p \in P: A(p) = p'} {\cal P}(p)$.
        \item Transitions: For each $z \in Z$, $A(z) = z'$, where $z' \in Z'$.
        For each vertex $v_z \in V_z$, $A(v_z) = v_z'$, where $v_z' \in V_z'$.
        For each $z \in Z$, where ${\cal
        Z}(z) = v_z$, $A(z) = z'$, and $A(v_z) = v_z'$, then ${\cal
        Z}'(z') = v_{z'}'$. 
        \item In Edges: For each edge $(v_z, v_x) \in E_{in}$, $A((v_z, v_x)) =
        (v_z', v_x')$, $A(v_x) = v_x'$, and $A(v_z) = v_z'$, where $(v_z',
        v_x')\in E_{in}'$;
        \item Out Edges: For each edge $(v_x, v_z) \in E_{out}$, $A((v_x, v_z))
        = (v_x', v_z')$; $A(v_x) = v_x'$, and $A(v_z) = v_z'$, where $(v_x',
        v_z')\in E_{out}'$;

        
        \item Transition Rates: For each $z' \in Z'$, ${\cal R}'({\bf p}', {\bf
        x}', z') = \sum\limits_{z \in Z: A(z)=z'} {\cal R}({\bf p}, {\bf
        x}, z)$.
    \end{itemize}
\end{definition}

The abstraction $(\Theta', \Omega')$ similarly defines the gradient $\nabla_{\Omega', \Theta'}({\bf p}', {\bf x}', t) = (\frac{dx_1'}{dt},
\frac{dx_2'}{dt}, \ldots)^T$, in terms of Equation \ref{eqn:flow}.
The abstraction thus expresses the gradient by aggregating terms from the
base Petrinet and semantics.  It preserves the flow on transitions, but
expresses the transition rates in terms of the base states.  As such, the
abstraction compresses the Petrinet graph structure, but at the cost of
expanding the expressions for transition rates. Moreover, the transition
rates refer to state variables and parameters that are not expressed
directly by the Petrinet and semantics, and by extension, the gradient. 

We modify the abstraction in what we call a \emph{bounded abstraction}, so that
it refers to the abstract, and not the base, Petrinet and semantics.  This
bounded abstraction replaces base elements with corresponding bounded elements.
For example, if $A(x_1) = x'$ and $A(x_2) = x'$ ($x_1$ and $x_2$ are base
variables represented by $x'$ in the abstraction), a possible transition rate
could be of the form
${\cal R}'({\bf p}', {\bf x}', z') = p_1 x_1 + p_2 x_2$.  By construction, we
know that $x_1 + x_2 = x'$.  However, in general $p_1 \not= p_2$, and we cannot
say that $p_1 x_1 + p_2 x_2 = p'x'$ for some $p'$.  Yet, if we replace $p_1$ and
$p_2$ by $p^{ub} = \max(p_1, p_2)$, then $p^{ub} x_1 + p^{ub} x_2 \geq p'x'$.  Simplifying, we
get $p^{ub} x_1 + p^{ub} x_2 = p^{ub}(x_1 + x_2) = p^{ub} x' \geq p'x'$.  A
similar argument can be made where $p^{lb} = \min(p_1, p_2)$ and we find that
$p^{lb} x' \leq p'x'$.  

By introducing the bounded parameters, we no longer
rely upon the base state variables or parameters.  However, in tracking the
effect of the bounded
parameters, the bounded abstraction must also track bounded rates and bounded
state variables.  The resulting bounded abstraction thus over-approximates the
abstraction and base model, wherein we can derive bounds on the state variables
at each time, which may correspond to a larger (hence over-approximation) set of
state trajectories.

\begin{definition}
A bounded abstraction $(\Theta^B, \Omega^B)$ of an abstraction $(\Theta',
\Omega')$ of $(\Theta, \Omega)$ replaces each element of $(\Theta',
\Omega')$ by a pair of elements denoting the lower and upper bound of that
element (and referred to with the ``$lb$'' and ``$ub$'' superscripts).  The
bounded abstraction defines:
\begin{itemize}
    \item State: For each $x' \in X'$,  $x^{lb}, x^{ub} \in X^B$.  For each
    $v_{x'}' \in V_x'$, ${\cal X}^B(x^{lb}) = v_{x^{lb}}^B$ and ${\cal X}^B(x^{ub}) =
    v_x^{ub}$.   For each $x^{lb}, x^{ub} \in X^B$, ${\cal I}^B(x^{lb}) = {\cal
    I}^B(x^{ub}) = {\cal I}'(x')$.
        \item Parameters: For each $p' \in P'$,
        let ${\cal P}^B(p^{lb}) = \min\limits_{p \in P: A(p) = p'} {\cal P}(p)$ and ${\cal P}^B(p^{ub}) = \max\limits_{p \in P: A(p) = p'} {\cal P}(p)$. 
        

        \item Transitions: For each $z' \in Z'$, $z^{lb}, z^{ub} \in Z^B$.
        For each vertex $v_z \in V_z$, $v_{z^{lb}}, v_{z^{ub}} \in V_z^B$.
        
        \item In Edges: For each edge $(v_{z'}, v_{x'}) \in E_{in}'$, $(v_{z^{lb}},
        v_{x^{lb}}), (v_{z^{ub}},
        v_{x^{ub}}) \in E^B_{in}$.
        \item Out Edges: For each edge $(v_{x'}, v_{z'}) \in E_{out}'$, $(v_{x^{ub}},
        v_{z^{lb}}), (v_{x^{lb}},
        v_{z^{ub}}) \in E^B_{out}$.

        
        \item Transition Rates: For each $z^{lb} \in Z^B$, ${\cal R}^B({\bf p}^B, {\bf
        x}^B, z^{lb}) = \min\limits_{z \in Z: A(z)=z'} {\cal R}({\bf p}, {\bf
        x}, z)$ (replacing ${\bf p}$ and $[\bf x]$ of the minimal rate by the 
        elements in ${\bf p}^B$ and ${\bf x}^B$ respectively, which minimize the
        rate), and ${\cal R}^B({\bf p}^B,
        {\bf
        x}^B, z^{ub}) = \max\limits_{z \in Z: A(z)=z'} {\cal R}({\bf p}, {\bf
        x}, z)$ (similarly replacing ${\bf p}$ and ${\bf x}$ of the maximal rate by the 
        elements in ${\bf p}^B$ and ${\bf x}^B$ respectively, which maximize the
        rate).
\end{itemize}
    
\end{definition}