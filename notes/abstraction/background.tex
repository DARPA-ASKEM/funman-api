\begin{definition}
    A Petrinet $\Omega$ is a directed graph $(V, E)$ with vertices $V=(V_x,
    V_z)$ partitioned into sets $V_x$ of state vertices and $V_z$ of transition
    vertices, and edges $E=(E_{in}, E_{out})$ partitioned into collections $E_{out}$ of
    flow-out and $E_{in}$ flow-in edges (relative to state vertices). 
\end{definition}



\begin{definition}
A flow-out edge $e \in E_{out}$ comprises a pair of vertices $(v_x,v_z)$, where
$v_x \in V_x$ is a state vertex, $v_z \in V_z$ is a transition vertex, and the
flow is directed from $v_x$ to $v_z$.  
\end{definition}

\begin{definition}
    A flow-in edge $e \in E_{in}$ comprises a pair of vertices $(v_z,v_x)$,
    similar to a flow-out edge, except that the flow is directed from $v_z$ to
    $v_x$.  
\end{definition}

\begin{example}
    The SIR model that stratifies the $S$ state variable into $S_1$ and $S_2$
    for two susceptible populations and defines $\Omega$ by:
    \begin{eqnarray*}
        V_x &=& \{v_{S_1}, v_{S_2}, v_{I}, v_{R}\}\\
        V_z &=& \{v_{inf_1}, v_{inf_2}, v_{rec}\}\\
        E_{in} &=& ((v_{inf_1}, v_{S_1}), (v_{inf_1}, v_{I}), (v_{inf_1}, v_{I}), (v_{inf_2}, v_{S_2}), (v_{inf_2}, v_{I}),
        (v_{inf_2}, v_{I}), (v_{rec}, v_{R}))\\
        E_{out} &=& ((v_{S_1}, v_{inf_1}), (v_{S_2}, v_{inf_2}),(v_{I}, v_{inf_1}), (v_{I}, v_{rec}))
    \end{eqnarray*}
\end{example}

\begin{definition}
    The ODE semantics $\Theta$ of the Petrinet $\Omega$ defines a tuple $(P, X,
    Z, {\cal I}, {\cal P}, {\cal X}, {\cal Z}, {\cal R})$ where 
    \begin{itemize}
        \item $P$ is a set of parameters;
        \item $X$ is a set of state variables;
        \item $Z$ is a set of transitions;
        \item ${\cal I}: S \rightarrow \reals$ assigns the initial value of
        state variables to a real number;
        \item ${\cal P}: P \rightarrow \reals \cup \reals \times \reals$ assigns
        parameters to a real number, or a pair of real numbers defining an
        interval;
        \item ${\cal X}: X \rightarrow V_x$ assigns state variables to state
        vertices;
        \item ${\cal Z}: Z \rightarrow V_z$ assigns transtions to transition
        vertices; and
        \item ${\cal R}: {\bf P} \times {\bf X} \times Z \rightarrow \reals$
        defines the rate of each transition  $z \in Z$ in terms of the set of
        parameter vectors ${\bf P}$ and state variable vectors ${\bf X}$.  
    \end{itemize}
    The elements of the Petrinet $\Omega$ and semantics $\Theta$ define the
    partial derivative $\frac{d {\bf x}}{dt}$, so that for each state variable
    $x \in X$:
    
    \begin{equation}\label{eqn:flow}
        \frac{dx}{dt} = \sum_{v_z \in V_z^{in(x)}} {\cal R}({\bf p}, {\bf x}, z) - \sum_{v_z \in V_z^{out(x)} } {\cal R}({\bf p}, {\bf x}, z)
    \end{equation}
\noindent where $V_z^{in(x)} = \{v_z \in V_z | (v_z, v_x) \in E_{in}\}$ and
    $V_z^{out(x)}=\{v_z \in V_z| (v_x, v_z) \in E_{out}\}$ are the transition
    vertices that flow in and out of the vertex $v_x$, respectively. We denote
    by $\nabla_{\Omega, \Theta}({\bf p}, {\bf x}, t) = (\frac{dx_1}{dt},
    \frac{dx_2}{dt}, \ldots)^T$, the gradient comprised of components in
    Equation \eqref{eqn:flow}.
\end{definition}

\begin{example}
    The stratified SIR model defines $\Theta$ by:
    \begin{eqnarray*}
        P &=& \{\beta_1, \beta_2, \gamma\}\\
        X &=& \{S_1, S_2, I, R\}\\
        Z &=& \{inf_1, inf_2, rec\}\\
        {\cal I} &=& \left\{ 
            \begin{array}{ll}
                0.45& :S_1\\
                0.45& :S_2\\
                0.1& :I\\
                0.0& :R
            \end{array}\right.\\
        {\cal P}&=& \left\{ 
            \begin{array}{ll}
                1e{-7}& :\beta_1\\
                2e{-7}& :\beta_2\\
                1e{-5}& :\gamma
            \end{array}\right.\\
            \\
        {\cal X} &=& \left\{ 
            \begin{array}{ll}
                v_{x} & : x \in X
            \end{array}\right.\\
        {\cal Z} &=& \left\{ 
            \begin{array}{ll}
                v_{z} & : z \in Z
            \end{array}\right.\\
        {\cal R} &=& \left\{ 
            \begin{array}{ll}
                \beta_1 S_1 I & : z_{inf_1}\\
                \beta_2 S_2 I & : z_{inf_2}\\
                \gamma I R & : z_{rec}\\
            \end{array}\right.\\
    \end{eqnarray*}
\end{example}


Using the partial derivatives defined by the Petrinet graph and semantics, we
can define the state vector at given time $t+dt$ with the forward Euler method
as:

\begin{eqnarray*}
    \frac{d {\bf x}}{dt} &=& \nabla_{\Omega, \Theta}({\bf p},{\bf x}, t)\\
    \frac{{\bf x}(t+dt)-{\bf x}(t)}{dt} &=& \nabla_{\Omega, \Theta}({\bf p},{\bf x},
    t)\\
    {\bf x}(t+dt)&=& \nabla_{\Omega, \Theta}({\bf p},{\bf x}, t)dt+ {\bf x}(t)
\end{eqnarray*}

