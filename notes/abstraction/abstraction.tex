\begin{definition}
    An abstraction $(\Theta', \Omega')$ of a Petrinet and the associated
    semantics $(\Theta, \Omega)$ that is produced by the abstraction operator
    $A$ has the following properties:
    \begin{itemize}
        \item State: For each $x \in X$,  $A(x) = x'$, where $x' \in
        X'$.  For each vertex $v_x \in V_x$,  $A(v_x) = v_x'$ where $v_x' \in
        V_x'$.   For each $x\in X$ where  ${\cal X}(x) =
        V_x$, $A(x) = x'$, and $A(v_x) = v_x'$, then ${\cal X}'(x')=
        v_{x'}'$.  For each $x' \in X'$, ${\cal X}'(x') = \sum\limits_{x \in X: A(x) = x'} {\cal X}(x)$.
        \item Parameters: For each $p \in P$, $A(p) = p'$, where $p'\in P'$.
        For each $p' \in P'$, ${\cal P}'(p') = \sum\limits_{p \in P: A(p) = p'} {\cal P}(p)$.
        \item Transitions: For each $z \in Z$, $A(z) = z'$, where $z' \in Z'$.
        For each vertex $v_z \in V_z$, $A(v_z) = v_z'$, where $v_z' \in V_z'$.
        For each $z \in Z$, where ${\cal
        Z}(z) = v_z$, $A(z) = z'$, and $A(v_z) = v_z'$, then ${\cal
        Z}'(z') = v_{z'}'$. 
        \item In Edges: For each edge $(v_z, v_x) \in E_{in}$, $A((v_z, v_x)) =
        (v_z', v_x')$, $A(v_x) = v_x'$, and $A(v_z) = v_z'$, where $(v_z',
        v_x')\in E_{in}'$;
        \item Out Edges: For each edge $(v_x, v_z) \in E_{out}$, $A((v_x, v_z))
        = (v_x', v_z')$; $A(v_x) = v_x'$, and $A(v_z) = v_z'$, where $(v_x',
        v_z')\in E_{out}'$;

        
        \item Transition Rates: For each $z' \in Z'$, ${\cal R}'({\bf p}', {\bf
        x}', z') = \sum\limits_{z \in Z: A(z)=z'} {\cal R}({\bf p}, {\bf
        x}, z)$.
    \end{itemize}
\end{definition}

\begin{example}
    The abstraction $(\Theta', \Omega')$ of the stratified SIR model defines
    (with the changed elements highlighted by ``*''):
    \begin{eqnarray*}
        A &=& \left\{ 
            \begin{array}{lll}
                S &: S_1 &*\\
                S &: S_2&*\\
                I &: I\\
                R &: R\\
               \beta &: \beta_1&*\\
               \beta &: \beta_2&*\\
               \gamma &: \gamma\\
               inf&: inf_1&*\\
               inf&: inf_2&*\\
               rec&: rec\\
               v_S &: v_{S_1}&*\\
               v_S &: v_{S_2}&*\\
               v_I &: v_{I}\\
               v_R &: v_{R}\\
               (v_{S}, v_{inf}) &: (v_{S_1}, v_{inf_1})&*\\
               (v_{S}, v_{inf}) &: (v_{S_2}, v_{inf_2})&*\\
               (v_{I}, v_{inf}) &: (v_{I}, v_{inf_1})&*\\
               (v_{I}, v_{inf}) &: (v_{I}, v_{inf_2})&*\\
               (v_I, v_{rec}) &: (v_I, v_{rec})\\
               (v_{inf}, v_I) &: (v_{inf_1}, v_I)&*\\
               (v_{inf}, v_I) &: (v_{inf_2}, v_I)&*\\
               (v_{rec}, v_R) &: (v_{rec}, v_R)\\
            \end{array}\right.\\
            {\cal R} &=& \left\{ 
            \begin{array}{lll}
                \beta_1 S_1 I +  \beta_2 S_2 I& : z_{inf}&*\\
                \gamma I R & : z_{rec}\\
            \end{array}\right.\\
    \end{eqnarray*}
\end{example}

The abstraction $(\Theta', \Omega')$ similarly defines the gradient $\nabla_{\Omega', \Theta'}({\bf p}', {\bf x}', t) = (\frac{dx_1'}{dt},
\frac{dx_2'}{dt}, \ldots)^T$, in terms of Equation \ref{eqn:flow}.
The abstraction thus expresses the gradient by aggregating terms from the
base Petrinet and semantics.  It preserves the flow on transitions, but
expresses the transition rates in terms of the base states.  As such, the
abstraction compresses the Petrinet graph structure, but at the cost of
expanding the expressions for transition rates. Moreover, the transition
rates refer to state variables and parameters that are not expressed
directly by the Petrinet and semantics, and by extension, the gradient. 

